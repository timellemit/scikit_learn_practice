\documentclass[a4paper,11pt]{article}
\usepackage[utf8]{inputenc}
\usepackage[english,russian]{babel}
\usepackage{bm} 



\title{\textbf{Курс <<Системы разработки данных и машинного обучения>>} \\
Домашнее задание №3}

\author{Национальный Исследовательский Университет
\\Высшая школа экономики}
\date{}
\begin{document}
\maketitle

\hrule
\noindent Авторы: Ю.С.~Кашницкий, А.В.~Шестаков, Д.И.~Игнатов \\
\hfill Срок сдачи: 08.06.14 23:59
\hrule


\section*{Задание 1 (10 баллов). Распознавание рукописных цифр с помощью методов классификации - $k$ ближайших соседей, логистическая регрессия и SVM}
Файл \verb"digits.mat" содержит признаковое описание изображений рукописных цифр $0$-$9$ и метки классов, соответствующие цифре (для цифры $0$ метка равна $10$). Для импорта mat-файла в рабочую область R можно пользоваться методом \verb"readMat" пакета \verb"R.matlab". Для чтения этого файла в словарь Python можно использовать команду \verb"loadmat" модуля \verb"scipy.io".

Признаковое описание множества примеров рукописной записи цифр задаётся матрицей $N \times 400$, где $N$ -- количество примеров. Каждый пример можно вывести на экран в виде изображения, преобразовав вектор-строку в матрицу размера $20\times20$ (команды \verb"reshape()"  и \verb"imagesc()" в \textsc{Matlab}'e).

В задании требуется:
\begin{enumerate}
  \item Для выбранного Вами языка программирования найти пакеты/модули, реализующие следующие алгоритмы классификации:
  \begin{enumerate}
    \item $k$ ближайших соседей ($k$ Nearest Neighbours);
    \item логистическая регрессия (Logistic regression);
    \item машина опорных векторов (Support Vector Machine, SVM).
  \end{enumerate}
  
  \item Разбить множество из $N$ объектов на обучающую и тестовую выборку в соотношении 80:20. 
  \item Для каждого алгоритма по отдельности использовать метод 4-кратного скользящего контроля (4-fold cross-validation) для определения параметров алгоритмов, при которых ошибка классификации минимальна. 
  \begin{enumerate}
    \item для метода $k$ ближайших соседей рассмотреть параметр $k$ из диапазона $\overline{3\dots6}$;
    \item для логистической регрессии в качестве параметра рассмотреть 10 разных коэффициентов регуляризации $\lambda$ (либо обратного коэффициента регуляризации $C = \frac{1}{\lambda}$ - зависит от конкретной реализации алгоритма) из определенного Вами диапазона;
    \item для SVM рассмотреть два вида ядра - полиномиальное и гауссово (RBF, radial-based function). В случае полиномиального ядра рассмотреть степени полинома 3 и 4. В случае гауссова ядра  - 10 коэффициентов $\gamma$ из определенного Вами диапазона. Также рассмотреть 10 разных обратных коэффициентов регуляризации ($C = \frac{1}{\lambda}$) из определенного Вами диапазона.
  \end{enumerate}
  Удобно пользоваться фунциями сеточной оценки параметров (если найдете). Например, в модуле \textsc{Scikit-learn} (язык Python) с помощью метода Grid Search можно запустить выбранный алгоритм, скажем, 200 раз для 20 значений одного параметра и 10 - другого. При этом лучшую пару параметров можно определить с помощью метода скользящего контроля.  

  \item Для оптимальных настроек трех алгоритмов:
  \begin{enumerate}
    \item Классифицировать объекты из тестовой выборки
    \item Определить <<победителей>> по следующим характеристикам:
		\begin{enumerate}
		\item Процент ошибок классификации
		\item Точность и полнота (precision \& recall)
		\item F-мера ($F = \frac{2*precision*recall}{precision+recall}$)
	  \end{enumerate}
  \end{enumerate}
\end{enumerate}

\end{document}